\begin{longtable}{|l|X|X|}
\caption{Minebench applications} \label{tab:minebench} \\

\hline 
\endfirsthead

\hline
\multicolumn{1}{|c|}{\textbf{Application}} & \multicolumn{1}{c|}{\textbf{Category}} & \multicolumn{1}{c|}{\textbf{Description}}\\
\hline 
\endhead

\hline \multicolumn{3}{|r|}{{Continued on next page}} \\ \hline
\endfoot

\hline
\endlastfoot

\multicolumn{1}{|c|}{\textbf{Application}} & \multicolumn{1}{c|}{\textbf{Category}} & \multicolumn{1}{c|}{\textbf{Description}}\\ \hline


ScalParC       & Classification          & Decision tree classification                                        \\ \hline
Naïve Bayesian & Classification          & Simple statistical classifier                                       \\ \hline
SNP            & Classification          & Hill-climbing search method for DNA dependency extraction           \\ \hline
Research       & Classification          & RNA sequence search using stochastic Context-Free Grammars          \\ \hline
SVM-RFE        & Classification          & Gene expression classifier using recursive feature elimination      \\ \hline
K-means        & Clustering              & Mean-based data partitioning method                                 \\ \hline
Fuzzy K-means  & Clustering              & Fuzzy logic-based data partitioning method                          \\ \hline
HOP            & Clustering              & Density-based grouping method                                       \\ \hline
BIRCH          & Clustering              & Hierarchical Clustering method                                      \\ \hline
Eclat          & Association Rule Mining & Vertical database, Lattice transversal techniques used              \\ \hline
Apriori        & Association Rule Mining & Horizontal database, level-wise mining based on Apriori property    \\ \hline
Utility        & Association Rule Mining & Utility-based association rule mining                               \\ \hline
GeneNet        & Structure Learning      & Learning Gene relationship extraction using microarray-based method \\ \hline
SEMPHY         & Structure Learning      & Learning Gene sequencing using phylogenetic tree-based method       \\ \hline
PLSA           & Optimization            & DNA sequence alignment using Smith-Waterman optimization method    
\end{longtable}